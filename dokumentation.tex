\documentclass{dsa}

\pagenumbering{gobble}

\usepackage{multicol}

\usepackage{listings}
\definecolor{mygreen}{rgb}{0,0.6,0}
\definecolor{mygray}{rgb}{0.5,0.5,0.5}
\definecolor{mymauve}{rgb}{0.58,0,0.82}

\lstset{ %
  basicstyle=\footnotesize\ttfamily, % the size of the fonts that are used for the code
  breakatwhitespace=false,           % sets if automatic breaks should only happen at whitespace
  breaklines=true,                   % sets automatic line breaking
  captionpos=b,                      % sets the caption-position to bottom
  commentstyle=\color{mygreen},      % comment style
  keepspaces=true,                   % keeps spaces in text, useful for keeping indentation of code (possibly needs columns=flexible)
  keywordstyle=\color{blue},         % keyword style
  language=[LaTeX]TeX,               % the language of the code
  numberstyle=\tiny\color{mygray},   % the style that is used for the line-numbers
  showspaces=false,                  % show spaces everywhere adding particular underscores; it overrides 'showstringspaces'
  showstringspaces=false,            % underline spaces within strings only
  showtabs=false,                    % show tabs within strings adding particular underscores
  stringstyle=\color{mymauve},       % string literal style
  tabsize=2,                         % sets default tabsize to 2 spaces
}

\sloppy

\begin{document}
   \begin{dsaTitlePage}
      \bfseries \Huge \LaTeX-Klasse für Das Schwarze Auge \\[15pt]
      \LARGE Für Charakterbögen, Abenteuer und Sonstiges \\[40pt]
      {\large \normalfont Felix Krause und Nico Weyand\\[10pt] \today \\[10pt] Version 0.2.1} \\[15pt]
      {\small \normalfont Diese Dokumentation ist lizensiert unter den Bedingungen der
       LaTeX Project Public License, Version 1.3c oder neuer.}
   \end{dsaTitlePage}
      
   \twocolumn[\section*{Überblick}]
   
   \subsection*{Einleitung}
   
   Dies ist eine \LaTeX-Klasse, die Umgebungen und Kommandos zur Verfügung
   stellt, um Dokumente im Stil der DSA-Regelbücher und Charakterbögen zu 
   erstellen - auf der Basis der Grafiken aus dem DSA-Fanpaket.
   
   \subsection*{Abhängigkeiten}
   
   Die Schriftart der Überschriften heißt \textit{Mason} und ist
   kostenpflichtig. Es gibt jedoch einen Klon namens \textit{Manson}, der
   kostenlos verfügbar ist (einfach im Internet suchen). Installiere zumindest
   die Variante \textit{Manson Bold} auf deinem System (auf Windows und OSX
   kannst du auf die heruntergeladene Schriftart doppelklicken und dort auf
   \textit{installieren} klicken). Für normale Schrift wird die freie Schriftart
   \textit{GaramondNo8} verwendet, die du \href{http://garamond.org}{hier}
   herunterladen kannst.
   
   Der Klasse liegen zwei Skripte bei, die das Fanpaket von Ulisses Spiele
   herunterladen und die Dateien darin umbenennen. Bevor du die Klasse benutzen
   kannst, musst du auf Windows das Skript mit der Endung \texttt{.ps1}, oder
   auf OSX und Linux das Skript mit der Endung \texttt{.sh} ausführen. Näheres
   steht in der \texttt{README.html}.
   
   Das Hintergrundbild für Charakterbögen ist nicht Bestandteil des Fanpakets.
   Standardmäßig erstellt die \LaTeX-Klasse Charakterbögen auf weißem Hintergrund
   (also auch ohne den Rand). Du kannst jedoch das Original-Hintergrundbild der
   DSA-Charakterbögen verwenden; eine Anleitung findest du im
   \texttt{README.html}. Hast die das Hintergrundbild in den Ordner 
   \texttt{fanpaket} gelegt, wird es automatisch als Hintergrundbild für deine
   Charakterbögen verwendet.
   
   Zum Erstellen der Dokumente wird eine TeX-Distribution benötigt, die
   \texttt{xelatex} beinhaltet (das trifft auf alle derzeit verbreiteten
   Distributionen zu). Außerdem müssen folgende LaTeX-Pakete verfügbar sein:
   \texttt{polyglossia, xunicode, fontspec, titlesec, xcolor, pgf, graphics
    wallpaper, environ}.
   
   Des Weiteren benötigst du \texttt{eforms} für editierbare Textfelder.
   Im Gegensatz zu den anderen Paketen musst du dieses manuell
   \href{http://www.ctan.org/pkg/eforms}{hier} herunterladen und dann in deine
   Distribution installieren (wie genau das geht, hängt von der Distribution ab).
   
   \subsection*{Mitgelieferte Dokumente}
   
   Der Klasse liegen bereits einige Dokumente bei:
   
   \begin{itemize}
      \item \textbf{dokumentation.pdf:} Dieses Dokument.
      \item \textbf{vertrautendokument.pdf:} Ein Charakterbogen für
            Vertrautentiere von Hexen und Geoden.
   \end{itemize}
   
   Wenn du das Vertrautendokument mit Hintergrundbild haben möchtest, kannst du
   es mit der beiliegenden Quelldatei \texttt{vertrautendokument.tex} selbst
   erstellen, nachdem du das Hintergrundbild in den \texttt{fanpaket}-Ordner
   gelegt hast.
   
   \subsection*{Dateimanagement}
   
   Üblicherweise werden \LaTeX-Pakete installiert, und von überall her
   verfügbar zu sein. Dies kannst du mit der DSA-Klasse ebenfalls machen;
   alternativ kannst du aber auch einfach \texttt{dsa.cls} und den Ordner
   \texttt{fanpaket} dorthin kopieren, wo auch dein Dokument liegt. Der
   Vorteil ist, dass du dann direkt Grafiken aus dem Fanpaket über relative
   Pfadangaben verwenden kannst.
   
   \begin{dsaBoxPortrait}
      \subsection*{\normalsize Informationen für Entwickler}
      
      Diese LaTeX-Klasse wird auf GitHub gehostet, das Repository befindet sich
      hier:
      
      \url{https://github.com/flyx/DSA-LaTeX}
      
      Wenn du den Code aus dem Repository aus\-checkst, statt das neueste
      Release zu nehmen, bist du auf dem neuesten Stand, allerdings hast du
      dann die Skripte zum Herunterladen des Fanpakets nicht, denn die werden
      erst beim Bauen eines Releases erzeugt. Die beiliegende \texttt{Makefile}
      nimmt dir die Arbeit ab, ein Release zu bauen. Um sie auszuführen, brauchst
      du \texttt{make}, \texttt{Python 2}, sowie die Python-Pakete
      \texttt{pyyaml}, \texttt{pystache} und \texttt{Markdown}. Ich teste das
      Makefile nicht auf Windows, aber man kann es dort vermutlich mit
      Cygwin zum Laufen kriegen. 
      
      Das Makefile ist nicht dafür gedacht, während der Entwicklung eingesetzt
      zu werden - es lädt bei jedem Aufruf das Fanpaket neu herunter!
      
   \end{dsaBoxPortrait}
   
   \begin{center}
      \includegraphics{./fanpaket/symbol-nandus.png}
   \end{center}
   
   \onecolumn
   
   \section*{Benutzung}
   
   \begin{multicols}{2}
      \subsection*{Grundsätzliches}
      
      Diese Klasse kann nur mit XeTeX benutzt werden. Das ist vor allem
      deshalb so, weil normales \TeX nicht die im System installierten
      TrueType-Schriftarten verwenden kann. Außerdem hat man mit XeTeX nicht
      das ständige Problem mit der mangelhaften Unicode-Unterstützung (im
      Deutschen vor allem wichtig für Umlaute). Entsprechend musst du deine
      Dokumente immer mit dem Kommando \texttt{xelatex} setzen. Wenn du einen
      grafischen \LaTeX-Editor verwendest, sollte dieser eine entsprechende
      Einstellung haben.
      
      \subsection*{Dokumentengerüst}
         
      Ein DSA-Dokument sieht etwa so aus:
      
      \begin{lstlisting}
\documentclass{dsa}
\begin{document}
   \begin{dsaTitlePage}
      \bfseries \Huge
      Mein tolles DSA-Dokument
   \end{dsaTitlePage}
   
   % Inhalt
\end{document}\end{lstlisting}
   
   \columnbreak
      
      Die \texttt{dsaTitlePage} ist für Charakterbögen natürlich nicht notwendig,
      ebensowenig wie die Überschriften. Das Erstellen von Charakterbögen wird
      auf der nächsten Seite beschrieben.
      
      \vspace{-10pt}
      \begin{center}
         \includegraphics{./fanpaket/symbol-aves.png}
      \end{center}
      \vspace{-15pt}
      
      \subsection*{Überschriften \& Layout}

      Überschriften werden, wie man in diesem Dokument sehen kann, in der
      Schriftart \textit{Manson} gesetzt. Standardmäßig wird ein zweispaltiges
      Layout verwendet. Eine Überschrift, die über die gesamte Seitenbreite
      geht, lässt sich folgendermaßen setzen:

      \begin{lstlisting}
\twocolumn[\section*{Überschrift}]\end{lstlisting}

      Dies startet eine neue Seite. Wenn - wie auf dieser Seite - Ein Kasten
      über die gesamte Breite der Seite gehen soll, muss die Seite selbst mit
      \verb|\onecolumn| begonnen werden. Die zweispaltigen Teile der Seite
      können dann mithilfe der \verb|multicols|-Umgebung erstellt werden. 
      
   \end{multicols}
   
   \begin{dsaBoxLandscape}
      \subsection*{Mit Grafiken hinterlegte Boxen}
      
      \begin{multicols}{2}
         Es ist drei Größen für Boxen verfügbar, die mit
         folgenden Umgebungen benutzt werden können:
         
         \begin{itemize} \itemsep0em
            \item \textbf{dsaBoxLandscape}: Querformat, gedacht für etwa eine
                  halbe Seite (so wie diese Box hier).
            \item \textbf{dsaBoxPortrait}: Hochformat, gedacht für etwa eine
                  Viertelseite (so wie die Box auf der vorherigen Seite).
            \item \textbf{dsaBoxLandscapeSmall}: Querformat, gedacht für etwa
                  eine Achtelseite.
         \end{itemize}
         
         Jede der Umgebungen nimmt als optionalen Parameter ihre Länge.
         Standardmäßig wird \verb|\linewidth| als Länge genommen. Die Höhe
         passt sich dem Inhalt an, das Hintergrundbild wird entsprechend
         verzerrt. Folgende vier Längenvariablen können geändert werden, um
         den Abstand des Inhalts zum Rand anzupassen:
         
         \begin{itemize} \itemsep0em
            \item \verb|\dsaBoxLeftPadding| (Standard: 12pt)
            \item \verb|\dsaBoxRightPadding| (Standard: 14pt)
            \item \verb|\dsaBoxTopPadding| (Standard: 7pt)
            \item \verb|\dsaBoxBottomPadding| (Standard: 14pt)
         \end{itemize}
         \vspace{-10pt}

      \columnbreak
      
         Standardmäßig werden die Boxen nicht in eine
         \texttt{float}-Umgebung gepackt. Das erscheint mir nicht als sinnvoll,
         denn die Boxen sollten schon genau dort erscheinen, wo der Autor sie
         haben möchte. Bildverweise oder ähnliches sind hier wohl eher fehl
         am Platz.
         
         Der Inhalt in den Boxen befindet sich automatisch in einer Minipage,
         das heißt, es können alle üblichen Formatierungstechniken verwendet
         werden. In dieser Box etwa wird der Text mittels einer
         \verb|multicols|-Umgebung zweispaltig gelayoutet.
         
         Beispiel:
         
         \begin{lstlisting}
\begin[0.5\linewidth]{dsaBoxPortrait}
   \subsection*{Informationskasten}
   
   Hier steht etwas interessantes drin.
\end{dsaBoxPortrait}\end{lstlisting}
         
      \end{multicols}
      \vspace{-8pt}
      
   \end{dsaBoxLandscape}
   
   \begin{dsaCharacterSheet}
      \section*{Charakterbögen}
      
      \begin{multicols}{2}
         \begin{dsaSheetBox}[8.5cm]
            \normalfont
            Die Umgebung \textbf{dsaCharacterSheet} startet eine neue Seite,
            auf der ein Charakterbogen, Handout oder ähnliches erstellt werden
            kann. Diese Seite wird mit dem Hintergrundbild \texttt{wallpaper.png}
            im \texttt{fanpaket}-Ordner hinterlegt, sofern dieses Bild vorhanden
            ist (wie auf der ersten Seite beschrieben, ist es nicht Bestandteil
            des Fanpakets). Ansonsten hat sie einen weißen Hintergrund.
            
            Die Seite hat weniger Rand als normale Seiten, um möglichst viel
            Platz verwenden zu können. Mit der Umgebung \textbf{dsaSheetBox}
            können schwarz umrandete Boxen wie diese hier erstellt werden.
            Wird das Hintergrundbild verwendet, so wird die Box mit
            halbtransparentem weiß gefüllt. \textbf{dsaSheetBox} nimmt als
            optionales Argument die Breite der Box, deren Standardwert
            \texttt{\textbackslash linewidth} ist.
            
            Auf der gesamten Seite wird standardmäßig die Schriftart für
            Überschriften verwendet, mit \texttt{\textbackslash normalfont} kann zur
            Schriftart für Fließtext gewechselt werden.
         \end{dsaSheetBox}
         
         \begin{dsaSheetBox}[8.5cm]
            \subsection*{Tabellen}
            
            \normalfont
            Charakterbögen beinhalten viele Tabellen. Hier sollte die
            \textbf{tabu}-Umgebung (statt \textbf{tabular}) verwendet werden.
            Hierfür wird folgendes Kommando zur Verfügung gestellt:
            
            \texttt{\textbackslash dsaRow\{<rowheight>\}\{<font>\}\{<content>\}}
            
            Fügt eine Zeile inklusive abschließendem Zeilenumbruch in eine Tabelle ein.
            Sowohl die Höhe der Zeile wie auch die benutzte Schriftart können angepasst
            werden. \texttt{<rowheight>} nimmt übliche Textgrößenparameter wie
            \texttt{\textbackslash scriptsize}. Mit \texttt{<font>} kann die
            Schriftart, -größe usw. für die gesamte Zeile festgelegt werden.
            
            Ein weiteres Kommando, das genutzt werden kann, ist:
            
            \texttt{\textbackslash dsaTextInput[<size>]\{<name>\}\{<width>\}}
            
            Es fügt ein editierbares Textfeld ein. \texttt{<size>} bestimmt die 
            Schriftgröße und steht standardmäßig auf 12 (ohne Einheit angeben!).
            \texttt{<name>} ist der Name des Textfelds (wird nirgends angezeigt,
            es muss nur darauf geachtet werden, dass kein Name doppelt benutzt
            wird). \texttt{<width>} setzt die Breite der Textbox.
         \end{dsaSheetBox}
      
         \columnbreak
         
         \begin{dsaSheetBox}[8.5cm]
            \subsection*{Beispiel: Zwei Boxen nebeneinander}
            \lstinputlisting{dokumentation-snippets/characterSheetBoxes.tex}
         \end{dsaSheetBox}
         
         \begin{dsaSheetBox}[8.5cm]
            \subsection*{Beispiel: Tabelle (Code)}
            \lstinputlisting{dokumentation-snippets/characterSheetTabu.tex}
         \end{dsaSheetBox}
         
         
         \begin{dsaSheetBox}[8.5cm]
            \subsection*{Beispiel: Tabelle (Output)}
            
            \renewcommand{\arraystretch}{1.4}
            \begin{tabu}{p{3.5cm}|p{1.1cm}|p{1.1cm}}
               \dsaRow{\scriptsize}{\scriptsize\normalfont\bfseries\centering}{& Start & Aktuell} \Xhline{2\arrayrulewidth}
               Eigenschaft 1 & \hspace{1pt} \dsaTextInput{E1orig}{0.75cm} & \cellcolor{white} \dsaTextInput{E1cur}{0.75cm} \\ \hline
               Eigenschaft 2 & \hspace{1pt} \dsaTextInput{E2orig}{0.75cm} & \cellcolor{white} \dsaTextInput{E2cur}{0.75cm} \\
            \end{tabu}
         \end{dsaSheetBox}
      \end{multicols}
      
   \end{dsaCharacterSheet}
\end{document}